%!TEX root = main.tex

\DeclareMathAlphabet{\mathcal}{OMS}{cmsy}{m}{n}  % Fix the awful mathcal fonts from mathptmx

\usepackage{amsthm,amsmath,amssymb,amsfonts}
\usepackage{xspace}
\usepackage{booktabs}
\usepackage{mathtools}
\usepackage[inline]{enumitem}
\usepackage{hyperref}
\usepackage{url}
\usepackage{multirow}
\usepackage{graphicx}
\usepackage{cite}
\usepackage{cryptocode}
\usepackage{setspace}
\usepackage{dashbox}
\usepackage{subcaption}
\usepackage{microtype}
\usepackage{etoolbox}

\usepackage{tgpagella}
\setlength{\oddsidemargin}{0in}
\setlength{\evensidemargin}{0in}
\setlength{\textwidth}{6.5in}
\setlength{\topmargin}{-.25in}
\setlength{\textheight}{8.5in}
\setlength{\parskip}{1em}
\setlength{\parindent}{0pt}

\graphicspath{ {figs/} }

\DeclarePairedDelimiter{\ceil}{\lceil}{\rceil}%
\DeclarePairedDelimiter{\floor}{\lfloor}{\rfloor}%
\DeclarePairedDelimiter\abs{\lvert}{\rvert}%
\DeclarePairedDelimiter\norm{\lVert}{\rVert}%
\DeclarePairedDelimiter\prns{(}{)}%
\DeclarePairedDelimiter\braces{\{}{\}}%
\DeclarePairedDelimiter\bracks{[}{]}%
\DeclarePairedDelimiterX\condprns[2]{(}{)}{\,#1 \;\delimsize\vert\; #2\,}
\DeclarePairedDelimiterX\condbrks[2]{[}{]}{\,#1 \;\delimsize\vert\; #2\,}
\DeclarePairedDelimiterX\condbraces[2]{\{}{\}}{\,#1 \;\delimsize\vert\; #2\,}
\DeclarePairedDelimiterX\cond[2]{}{}{#1 \;\delimsize\vert\; #2}

\newcommand{\bE}{\mathbb{E}}
\newcommand{\bP}{\mathbb{P}}
\newcommand{\bI}{\mathbb{I}}
\newcommand{\bR}{\mathbb{R}}
\newcommand{\bB}{\mathbb{B}}
\newcommand{\R}[1]{\mathbb{R}^{#1}}
\newcommand{\Rn}{\mathbb{R}^n}
\newcommand{\sZ}{\mathcal{Z}}
\newcommand{\sX}{\mathcal{X}}
\newcommand{\sV}{\mathcal{V}}
\newcommand{\sA}{\mathcal{A}}
\newcommand{\sB}{\mathcal{B}}
\newcommand{\sC}{\mathcal{C}}
\newcommand{\sD}{\mathcal{D}}
\newcommand{\sE}{\mathcal{E}}
\newcommand{\sG}{\mathcal{G}}
\newcommand{\sH}{\mathcal{H}}
\newcommand{\sL}{\mathcal{L}}
\newcommand{\sN}{\mathcal{N}}
\newcommand{\sY}{\mathcal{Y}}
\newcommand{\sF}{\mathcal{F}}
\newcommand{\sO}{\mathcal{O}}
\newcommand{\sT}{\mathcal{T}}
\newcommand{\rad}{\mathscr{R}}

\newcommand{\inv}[1]{{#1}^{-1}}
\newcommand{\trn}[1]{{#1}^{\intercal}}
\newcommand{\indep}{\;\rotatebox[origin=c]{90}{$\models$}\;}
\newcommand{\sample}{\leftarrow_\$}
\newcommand{\bin}{\braces{0,1}}


\theoremstyle{plain}
\newtheorem{theorem}{Theorem}
\newtheorem{lemma}[theorem]{Lemma}
\newtheorem{proposition}[theorem]{Proposition}
\newtheorem{corollary}[theorem]{Corollary}
\theoremstyle{definition}
\newtheorem{assumption}[theorem]{Assumption}
\newtheorem{definition}[theorem]{Definition}
\newtheorem{remark}[theorem]{Remark}
\newtheorem{example}[theorem]{Example}
\newenvironment{sketch}{\renewcommand{\proofname}{Proof Sketch}\renewcommand{\qedsymbol}{}\proof}{\endproof}

\newtheorem{manualtheoreminner}{Corollary}
\newenvironment{manualtheorem}[1]{%
  \renewcommand\themanualtheoreminner{#1}%
  \manualtheoreminner
}{\endmanualtheoreminner}

\DeclareMathOperator*{\argmax}{argmax}
\DeclareMathOperator*{\argmin}{argmin}
\DeclareMathOperator*{\prob}{\bP}
\DeclareMathOperator*{\ex}{\bE}

\newcommand{\hlchange}[1]{{\setlength{\fboxsep}{0pt}\colorbox{pink}{$\displaystyle#1$}}}
